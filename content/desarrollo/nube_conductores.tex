\section{Nube de Conductores}\label{section:NubeConductores}
El núcleo del sistema es una aplicación desplegada en la nube, la cual se ha denominado \emph{Nube de Conductores}, donde se concentran en una base de datos la información relativa a ciclistas y vehículos a motor. Un servicio de aplicación web embebido llamado \emph{Jetty} se encarga de recibir y atender los mensajes \Gls{http/1.1} que son enviados desde la parte de vehículos a motor y ciclistas. Dos \emph{Handler} independientes se encargan de filtrar los mensajes que no han sido correctamente construidos, es decir, tienen un formato inválido, e insertar y actualizar los datos de la base de datos.

Para el despliegue de la aplicación se utilizado una máquina virtual \emph{Ubuntu Server 14.04 LTS} que cuenta con 2048 MiB de memoria RAM y 2 n\'ucleos para procesamiento. También se ha reservado un dominio público para que las peticiones puedan ser enviadas al servidor. Gracias a la herramienta \emph{ANT} se puede cambiar fácilmente la plataforma donde se distribuya la aplicación, además esta configurada para poder ser ejecutada directamente con el comando \emph{run}.

% AÑADIR UN DIAGRAMA DE CLASES EXPLICATIVA DE LA COMUNICACIÓN ENTRANTE Y SALIENTE DE LA NUBE

\subsection{Procesos}\label{ssection:procesos}
A través del API de \emph{Jetty} la aplicación crea un servidor con dos manejadores de mensajes, uno para ciclistas y otro para vehículos a motor. A través de ellos la \emph{Nube de Conductores} recibe datos de ciclistas y vehículos a motor, almacenándolos en una base de datos interna sin necesidad de utilizar un DBMS, ya que no hace falta que los datos sean persistentes más tiempo de lo que los vehículos estén emitiendo su posición. Cada manejador posee un ThreadPool con el que crea un gestor para cada mensaje recibido, este esta limitado a un número de hilos para evitar que la aplicación se colapse. 

Un registro se considera antiguo cuando no ha sido refrescado en un período de un minuto. Para evitar que emplee información obsoleta, se ejecuta una rutina que tan solo mantiene en memoria los registros que periódicamente están siendo actualizados; esto se realiza gracias al campo de \emph{timestamp}.

Paralelamente, otro algoritmo compara las posiciones de los vehículos. Cuando se detecta que los vehículos a motor y los ciclistas están próximos - en un rango menor a 200 metros - se manda a ambos vehículos una alerta avisándoles de su proximidad \emph{[Algoritmo \ref{alg:proximidadVehiculos}]}.

\begin{listing}
	\begin{minipage}{.4\textwidth}
		\begin{minted}[linenos=true]{java}
for (Motorist m : lMotorist) {
  for (Cyclist c : lCyclist) {
    if (isCollisionDanger(m, c)) {
      sendWarningToMotorist(c);
      sendCyclistPositionToMotorist(c);
    }
  }
}
		\end{minted}
	\end{minipage}
	\caption{Cálculo de la proximidad de los vehículos}\label{alg:proximidadVehiculos}
\end{listing}