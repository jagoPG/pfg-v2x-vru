\section{Tecnología}
En esta sección se va a analizar las herramientas y tecnologías empleadas en
el proyecto. La sección ha sido dividida entre software y hardware debido a que
ambas presentan gran importancia en el proyecto.

\subsection{Software}

\subsection{Hardware}
El hardware empleado se puede dividir en dos grupos: por un lado el hardware
empleado por el \gls{obu} y \gls{rsu}, y por el otro el casco \gls{ble} que
visten los ciclistas para saber en qué dirección se aproxima el vehículo.

El hardware que integran los vehículos y la infraestructura de las carreteras
se trata de la plataforma experimental NEC Linkbird-MX. Es una minicomputador
que posee una tarjeta de red que le permite comunicarse con otros dispositivos
Linkbird-MX a través de una red \Gls{802.11p}; a esta tarjeta de red se puede
conectar dos antenas. Requiere de un \gls{gps} \gls{usb} para recibir la
posición del vehículo, aunque esta característica es opcional.

El casco \gls{ble} posee un microcontrolador \gls{SoC} \emph{ Texas Instruments
CC2540} para realizar aplicaciones de bajo consumo empleando Bluetooth. Este
dispositivo de bajo presupuesto permite realizar nodos maestros o esclavos
a través de \gls{ble}. Este dispositivo permite flashear programas de hasta
128 KB, suficiente para hacer la aplicación que se requiere en este proyecto.
Aunque no se encuentre soportado oficialmente, este dispositivo tiene varios
puertos \gls{gpio}, los cuales se van a emplear para conectar los led. Estos
microcontroladores se emplean para aplicaciones donde no se requiera gran
capacidad de computo, y se necesite que exista una buena autonomía.
