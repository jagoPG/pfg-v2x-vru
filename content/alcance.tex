\chapter{Alcance y objetivos}\label{cha:alcance}
La mejora en la miniaturización de la tecnología, y la calidad de los servicios de comunicación el concepto de ciudad inteligente es más posible de lograr que nunca. Este proyecto propone el diseño, desarrollo y validación de un escenario de movilidad cooperativa, segura y eficiente en la cual los usuarios (conductores, pasajeros y ciclistas) dispongan de una serie de aplicaciones, que proporcionen información y reciban información; la cual es procesada en tiempo real. Son tres los retos que afronta este proyecto:

\begin{enumerate}
	\item Diseñar una infraestructura de comunicaciones dinámica, y que integre diferentes tecnologías de comunicaciones de corto (\gls{802.11p}) y largo alcance (\gls{lte}).

	\item Procesar la información generada por todos los usuarios y agentes en la infraestructura.

	\item Promover la participación de todos los usuarios.

\end{enumerate}
Para lograrlo, este proyecto combina información centralizada en la nube y tecnologías de computación distribuida con tecnologías de comunicaciones, donde las comunicaciones \gls{v2x} y las comunicaciones móviles son utilizadas indistintamente para proporcionar información relevante a todos los usuarios. Gracias a ello, este proyecto favorecerá un entorno donde los vehículos, Smartphone y/o Tablet podrán operar entre ellas para proveer servicios de movilidad a los usuarios.

Este proyecto, basándose en la arquitectura \gls{its} definida por ITS EN 302 665, integrará información procedente de elementos de infraestructura, vehículos, terminales móviles con servicios desplegados en la nube y ofrecidos a los usuarios, en una red colaborativa capaz de diferenciar necesidades individuales y globales. Para ello, los datos recogidos serán procesados por algoritmos preparados para procesar grandes volúmenes de información, y que tendrán en cuenta los efectos globales de sus acciones, para lo cual se dispondrá de enlace de control para regular los sistemas de un modo descentralizado. Por tanto, el sistema deberá responder dinámicamente las necesidades de los usuarios.

Se ha seleccionado la tecnología \gls{lte} ya que se ha establecido como la siguiente generación de comunicaciones móviles, disponible masivamente y favorecerá el despliegue de \gls{its}. De este modo, la integración de tecnologías de corto alcance como \Gls{802.11p} con \gls{lte} es obligatorio para la rápida adopción de aplicaciones para la movilidad.

Para validar este proyecto, se implementarán las siguientes aplicaciones:
\begin{itemize}
	\item Intersecciones seguras: los vehículos próximos a una intersección intercambiarán su destino en la misma, avisando de su presencia para evitar accidentes.

	\item Navegación segura: los usuarios vulnerables de la carretera podrán reportar su posición a los vehículos que les rodeen y al mismo tiempo ser alertados de vehículos que se aproximen.

	\item Conducción colaborativa: se desplegarán una serie de aplicaciones destinadas a mejorar la eficiencia de las infraestructuras y su seguridad gracias a los datos intercambiados entre todos los usuarios y elementos fijos: frenada de emergencia, cambio de carril, velocidad adaptativa, etc.
\end{itemize}

El principal objetivo de este trabajo es el desarrollo de aplicaciones para hacer posible la comunicación entre vehículos y ciclistas, a través de redes vehiculares y móviles. Se desea que estos dos agentes de la carretera posean información para poder conocer la posición relativa de los agentes a su alrededor y así poder evitar accidentes. Las aplicaciones desarrolladas serán desplegadas por un lado en los sistemas embarcados de los vehículos y la infraestructura en la carretera, en los móviles inteligentes de los ciclistas y en la nube, donde un servidor central permitirá la comunicación entre diferentes plataformas. Se busca la creación de una plataforma lo más abierta posible a diferentes tecnologías, así como el uso de software libre para su desarrollo.

La verificación del proyecto se hará a través de las pruebas unitarias que se han creado para probar el sistema, las pruebas que se han realizado en la calle y a las conclusiones que se han llegado tras analizar las mediciones obtenidas. Se desea verificar la calidad de las comunicaciones entre los diferentes agentes de la carretera, así como la precisión de los algoritmos desarrollados para prever los accidentes y obtener información a través de los datos recogidos por los mensajes vehiculares. Para saber más sobre cómo se han verificado los resultados del proyecto ir al capítulo \ref{cha:pruebas}.

Las aplicaciones que serán desarrolladas deben poseer una serie de características:
\begin{itemize}
	\item Universalidad: el sistema a desarrollar debe ser flexible a diferentes tecnologías. Actualmente existen diversos fabricantes que aún no cumplen los estándares que se están estableciendo, por lo que se debe permitir que la adaptación de los diferentes sistemas sea lo más sencillo posible.
	
	\item Interfaz gráfica: la interacción del usuario con las aplicaciones debe ser sencilla, que no requiera de ningún conocimiento técnico para realizar sus funciones.
	
	\item Localización de agentes en carretera: los vehículos y ciclistas intercambian información sobre sus posiciones a través de mensajes \gls{cam}.
	
	\item Predicción de accidentes: los usuarios deben poder percibir las situaciones de peligro. En el caso de los vehículos a motor se reproduce una alarma sonora, y en el de los ciclistas se enciende un led que llevan colocado en el casco, además de mostrarse la alerta en el Smartphone; en caso de tenerlo a la vista.
\end{itemize}
