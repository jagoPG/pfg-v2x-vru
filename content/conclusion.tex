\chapter{Conclusiones y líneas futuras}
En este proyecto se ha presentado una plataforma para aumentar la seguridad de los \gls{vru} en el dominio de una ciudad inteligente. Esta solución convive perfectamente en este tipo de entornos, ya que está diseñado para adaptarse a cualquier arquitectura de comunicación. En este caso, los vehículos se comunican a través de redes \gls{802.11p} con la infraestructura de la carretera, proveyendo de sus posiciones. Por otro lado, los ciclistas - los cuales son considerados \gls{vru} - se comunican a través de redes móviles con la plataforma central, la cual recibe las posiciones de estos ciclistas. Gracias a la información que recibe de ambos extremos, la plataforma puede predecir qué vehículos pueden encontrarse en una situación de peligro y avisar a ambos extremos.

Gracias a las simulaciones realizadas se conoce el rendimiento que podría tener la plataforma en un despliegue real. Estos resultados sirven para poder diseñar la solución más adaptada a las tecnologías que hoy en día tenemos disponibles, además de proveer de una base sobre la que poder trabajar. 

El proyecto ha resultado exitoso ya que cumple con todos los objetivos y requisitos que se planificaron. Ha sido validada satisfactoriamente a pequeña escala en un entorno real, lo cual nos indica que esta solución puede tener futuro en una ciudad inteligente. El único problema detectado ha sido que la posición de los dispositivos móviles no es precisa, esto puede mejorar en el futuro cercano, ya que a grandes pasos el hardware de los dispositivos móviles evoluciona y es mejorado. 

La solución aportada está diseñada para soportar la escalabilidad, y poder incluir nuevas funciones en todas las aplicaciones. Tanto los protocolos empleados como los formatos que se han implementado permiten una amplia variedad de nuevas funciones; entre otras, actualmente se está trabajando en implementar streaming de vídeo y audio en tiempo real sobre redes \gls{802.11p}.

En el futuro, se tendría que encontrar un método para lograr obtener posiciones más fiables de los dispositivos móviles, ya que los \gls{gps} que emplean los smartphone actuales no proveen de posiciones suficientemente precisas como para garantizar que las predicciones realizadas por la plataforma no den falsos positivos. El \gls{hmi} del ciclista y de los vehículos también tendrá que ser mejorado para ofrecer mayor información, tanto del entorno como sobre la propia seguridad: emplear otros elementos wearables para notificar a los ciclistas de la presencia de los vehículos de manera más eficiente (por ejemplo mediante vibración), estado de la carretera, tramos en obras, etc.