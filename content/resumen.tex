\chapter*{Resumen}
El desarrollo y despliegue de sistemas inteligentes en la carretera (ITS) trae consigo un incremento de la seguridad y una mayor rapidez a la hora de responder a eventos de emergencia. Este tipo de tecnologías han demostrado ser lo suficientemente robustas para utilizarse como apoyo para lidiar con la gestión de tráfico, mejora de la seguridad vial... y actualmente están apareciendo vehículos que además de tener un propio sistema de auto-diagnostico, pueden interactuar con su entorno.

\vspace{2em}

{\Large\bfseries\sectionfont Descriptores}
\vspace{3\medskipamount}

redes vehiculares, agentes vulnerables, ciudades inteligentes.

\cleardoublepage\tableofcontents
\cleardoublepage\listoffigures
\cleardoublepage\listoftables
\cleardoublepage\listoflistings

\mainmatter
\pagestyle{phdthesis}