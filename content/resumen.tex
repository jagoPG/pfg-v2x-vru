\chapter*{Resumen}
Este proyecto tiene el objetivo de desarrollar servicios de movilidad cooperativa, a través de
redes híbridas, para mejorar la seguridad a los \gls{vru} proporcionando la información necesaria
a los vehículos que compartan el mismo medio. Para recoger la información relevante, se debe combinar
información centralizada en la nube y tecnologías distribuidas. Para la comunicación entre usuarios se
ha planeado emplear redes de corto alcance (\Gls{802.11p}) y largo alcance (\gls{lte}),
ya que son las que permiten una rápida adopción de las aplicaciones de movilidad.

Para la realización del proyecto, se requiere diseñar y desarrollar varias aplicaciones: una plataforma
en la nube que permita la comunicación entre Smartphone y unidades vehiculares, así como algoritmos para
la detección de situaciones peligrosas aquellas en las que los vehículos puedan colisionar) y avisar a
los usuarios involucrados. Así mismo, se requiere de aplicaciones en todos los actores involucrados en
el proyecto (vehículos, ciclistas y unidades desplegadas en carretera).

Para verificar el sistema se deben realizar pruebas en un entorno controlado, en el cual debe pasar un
plan de pruebas definido: comunicación eficiente entre los usuarios, facilidad de uso, rendimiento de
la plataforma...

% VERSION ANTIGUA
%Desde que se ha conseguido reducir el tamaño de los computadores, junto con el precio que cuesta producirlos, se ha
%incrementado un aumento de la presencia de sistemas informáticos en nuestro día a día. Así ha surgido la computación
%ubicua, la cual ha demostrado ser ya una realidad y busca hacernos la vida más sencilla. Uno de los mayores ejemplo de
%la madurez de esta tecnología es la ciudad de New Songdo (Corea del Sur), una metrópoli la cual tiene desplegada sistemas
%de información en los aparcamientos, cines, viviendas... los cuales interactúan con las personas a través de tarjetas.

%Uno de los áreas aún en investigación de los sistemas ubicuos es el desarrollo y despliegue de sistemas inteligentes en la
%carretera (ITS). Éstos persiguen el incrementar la seguridad de todos los agentes de la carretera y ofrecer una mayor rapidez
%a la hora de responder a eventos de emergencia. Este tipo de tecnologías han demostrado ser lo suficientemente robustas
%para utilizarse como apoyo para lidiar con la gestión de tráfico, mejora de la seguridad vial, aportar información al conductor...
%Actualmente están apareciendo vehículos que además de tener un propio sistema de auto-diagnostico, son capaces de
%interactuar con su entorno.

\vspace{2em}

{\Large\bfseries\sectionfont Descriptores}
\vspace{3\medskipamount}

redes vehiculares, agentes vulnerables, ciudades inteligentes.

\cleardoublepage\tableofcontents
\cleardoublepage\listoffigures
\cleardoublepage\listoftables
\cleardoublepage\listoflistings

\mainmatter
\pagestyle{phdthesis}
