\chapter*{Resumen}
Desde que se ha conseguido reducir el tamaño de los computadores, junto con el precio que cuesta producirlos, se ha incrementado un aumento de la presencia de sistemas informáticos en nuestro día a día. Así ha surgido la computación ubicua, la cual ha demostrado ser ya una realidad y busca hacernos la vida más sencilla. Uno de los mayores ejemplo de la madurez de esta tecnología es la ciudad de New Songdo (Corea del Sur), una metrópoli la cual tiene desplegada sistemas de información en los aparcamientos, cines, viviendas... los cuales interactúan con las personas a través de tarjetas.

Uno de los áreas aún en investigación de los sistemas ubicuos es el desarrollo y despliegue de sistemas inteligentes en la carretera (ITS). Éstos persiguen el incrementar la seguridad de todos los agentes de la carretera y ofrecer una mayor rapidez a la hora de responder a eventos de emergencia. Este tipo de tecnologías han demostrado ser lo suficientemente robustas para utilizarse como apoyo para lidiar con la gestión de tráfico, mejora de la seguridad vial, aportar información al conductor... Actualmente están apareciendo vehículos que además de tener un propio sistema de auto-diagnostico, son capaces de interactuar con su entorno.

\vspace{2em}

{\Large\bfseries\sectionfont Descriptores}
\vspace{3\medskipamount}

redes vehiculares, agentes vulnerables, ciudades inteligentes.

\cleardoublepage\tableofcontents
\cleardoublepage\listoffigures
\cleardoublepage\listoftables
\cleardoublepage\listoflistings

\mainmatter
\pagestyle{phdthesis}