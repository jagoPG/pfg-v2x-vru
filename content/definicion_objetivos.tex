\chapter{Definición de objetivos}
El principal objetivo de este trabajo es el desarrollo de aplicaciones para hacer
posible la comunicación entre vehículos y ciclistas, a través de redes vehiculares
y móviles. Se desea que estos dos agentes de la carretera posean información para
poder conocer la posición de los agentes a su alrededor y así poder evitar accidentes.
Las aplicaciones desarrolladas serán desplegadas por un lado en los sistemas
informáticos de los vehículos y la infraestructura en la carretera, en los móviles
inteligentes de los ciclistas y en la nube, donde un servidor central permitirá la
comunicación entre diferentes plataformas. Se busca la creación de una plataforma lo
más abierta posible a diferentes tecnologías, así como el uso de software libre para
su desarrollo.

Los resultados que permitan verificar el sistema se harán a través de las pruebas
unitarias que se han creado para probar el sistema, las pruebas que se han realizado
en la calle y a las conclusiones que se han llegado tras analizar las mediciones
obtenidas. Se desea verificar la calidad de las comunicaciones entre los diferentes
agentes de la carretera, así como la precisión de los algoritmos desarrollados para
prever los accidentes y obtener información a través de los datos recogidos por los
mensajes vehiculares. Para saber más sobre cómo se han verificado los resultados del
proyecto ir al capítulo \ref{cha:pruebas}.

Las aplicaciones que serán desarrolladas deben poseer una serie de características:
\begin{itemize}
	\item Universalidad: el sistema a desarrollar debe ser flexible a diferentes
	tecnologías. Actualmente existen diversos fabricantes que aún no cumplen los
	estándares que se están estableciendo, por lo que se debe permitir que la
	adaptación de los diferentes sistemas sea lo más sencillo posible.

	\item Interfaz gráfica: la interacción del usuario con las aplicaciones debe
	ser sencilla, que no requiera de ningún conocimiento técnico para realizar
	sus funciones.

	\item Localización de agentes en carretera: los vehículos y ciclistas
	intercambian información sobre sus posiciones a través de mensajes CAM.

	\item Predicción de accidentes: los usuarios deben poder percibir las situaciones
	de peligro. En el caso de los vehículos a motor se reproduce una alarma sonora, y
	en el de los ciclistas se enciende un LED que llevan colocado en el casco, además
	de mostrarse la alerta en el Smartphone; en caso de tenerlo a la vista.
\end{itemize}
