\chapter{Producto final}
El producto final se divide en tres grupos de aplicaciones que están presentes en diferentes
agentes de la carretera. Cada grupo de aplicaciones está formado por software y hardware
y constituye una unidad desde el punto de vista de usuario.

El primer grupo se trata del sistema en la nube que se encarga de comunicar el lado de
vehículos a motor y ciclistas. Además, detecta situaciones de peligro y pone en aviso a
cada agente implicado. Estas son las funciones que debe cumplir:
\begin{itemize}
	\item \textbf{Aplicación web:} la nube debe poder accederse a través de una aplicación
	web, empleando servlets para recibir y procesar la información.
	
	\item \textbf{Base de datos:} se deben poder almacenar datos de las peticiones enviadas
	por los vehículos. No es necesario que los datos sean persistentes, cuando la aplicación
	se cierra los datos deben ser eliminados.
	
	\item \textbf{Algoritmos para la detección de colisiones:} se implantarán algoritmos para detectar
	la proximidad entre dos vehículos, para actuar en consecuencia.
	
	\item \textbf{Sistema de avisos:} debe poder enviar mensajes a diferentes plataformas a
	través de redes móviles.
	
	\item \textbf{Sistema cloud:} la aplicación debe estar preparada para ser desplegada y
	ejecutada en la nube.
\end{itemize}

El segundo grupo se trata de todas las aplicaciones encargadas de obtener información en la
red vehicular, y recibir la información generada en la nube: \gls{rsu}, \gls{obu} y \gls{hmi}. Por
una parte, el \gls{obu} debe cumplir los siguientes requisitos:
\begin{itemize}
	\item \textbf{Comunicación vehicular:} debe poder comunicarse con otros dispositivos que
	componen la red vehicular.
	
	\item \textbf{Sistema de navegación:} debe implementar un sistema de navegación GPS para
	poder obtener las posiciones del vehículo en todo momento. Se requiere que estos datos sean
	altamente precisos.
	
	\item \textbf{Canal para sincronizar dispositivos:} debe ser posible emparejar un dispositivo
	móvil para mostrar la información obtenida al usuario.
\end{itemize}

Los requisitos que debe cumplir el \gls{hmi} son:
\begin{itemize}
	\item \textbf{Canal para sincronizar dispositivos:} el dispositivo debe poder emparejarse a la
	infraestructura del vehículo. Preferiblemente se desea que de forma inalámbrica.
	
	\item \textbf{Universalidad:} la aplicación a desarrollar debe ser desplegada al mayor número
	de usuarios posible.
	
	\item \textbf{No debe ser una distracción:} emplear el menor número de elementos dinámicos
	posibles, ya que al ser usado durante la conducción no debe distraer al conductor. Emplear
	forma visual y sonora para notificar al usuario de los eventos en carretera.
\end{itemize}

Por otra parte los requisitos del \gls{rsu} son:
\begin{itemize}
	\item \textbf{Comunicación vehicular:} debe poder comunicarse con otros dispositivos que
	componen la red vehicular.
	
	\item \textbf{Comunicación móvil:} debe tener acceso a Internet a través de una red móvil.
	
	\item \textbf{Sistema de retransmisión:} debe poder retransmitir los mensajes que recibe de
	la red de conductores, con la menor latencia posible.
\end{itemize}

Finalmente, el último grupo es la aplicación de ciclistas y servidor \gls{gcm} que provee
notificaciones a los ciclistas sobre los eventos en carretera, y al mismo tiempo permite a
los ciclistas mandar información sobre su posición. Además del casco \glossary{ble} que 
permite notificar al ciclista de un peligro inminente a través de leds instalados en el casco
de seguridad. Deben cumplir:
\begin{itemize}
	\item \textbf{Comunicación con la nube:} debe tener acceso a Internet a través de una red
	móvil para poder transmitir a la nube la información sobre la posición del ciclista.
	
	\item \textbf{Sistema de navegación:} debe implementar un sistema de navegación GPS para
	poder obtener las posiciones del ciclista en todo momento. Se requiere que estos datos sean
	lo más precisos posibles sin un alto consumo de batería.
	
	\item \textbf{No debe ser una distracción:} emplear el menor número de elementos dinámicos
	posibles, ya que al ser usado durante la conducción no debe distraer al ciclista. Emplear
	forma visual y sonora para notificar al usuario de los eventos en carretera.
	
	\item \textbf{Salidas en individual y en grupo:} debe poder usarse para salidas de ciclismo
	de forma individual y en grupo, sin necesidad de conocimientos técnicos para poder emplear
	la aplicación con soltura.
	
	\item \textbf{Emparejamiento:} debe poder emparejarse la aplicación con el casco de ciclistas
	a través de tecnología \gls{ble}.
\end{itemize}

Así mismo, a través de las pruebas realizadas, los resultados obtenidos y las conclusiones
generadas, se puede generar diferentes estudios que motiven nuevos desarrollos dentro del
dominio de los \gls{vru}:
\begin{itemize}
	\item \textbf{Rendimiento de módulos NEC Linkbird MX}.
	
	\item \textbf{Empleo de comunicaciones \gls{wave} en entornos urbanos}.
	
	\item \textbf{Eficiencia de algoritmos para la predicción de colisiones.}
	
	\item \textbf{Conclusiones de la implementación de un sistema en la nube
		en un entorno vehicular.}
\end{itemize}
