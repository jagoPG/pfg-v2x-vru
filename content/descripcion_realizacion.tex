\chapter{Descripción de la realización}
El equipo del proyecto está formado por el director del proyecto y dos
desarrolladores de software; en la Figura \ref{fig:organigrama} se puede
observar el organigrama del proyecto. El director del proyecto es a su vez el
cliente de la solución que se va a desarrollar, por lo que es el que se
encargará de validar cada componente principal de la solución; una vez que ha
sido correctamente depurado. Los desarrolladores son los encargados del diseño,
programación, despliegue, pruebas y documentación del proyecto; la terminación
de sus tareas reside en la decisión del director del proyecto una vez que ha
comprobado que cumple los requisitos acordados.

\begin{figure}[H]
	\begin{center}
		\includegraphics[scale=0.5]{organigrama}
		\caption{Organigrama del proyecto}
		\label{fig:organigrama}
	\end{center}
\end{figure}

La planificación del proyecto se ha desarrollado para poder mantener un ritmo
continuo de desarrollo de tipo ágil, y conseguir un aprovechamiento máximo del
proyecto. Durante el desarrollo de este proyecto se han realizado paralelamente
pequeños proyectos, por lo que es  necesario que se puedan observar los
resultados del desarrollo de la forma más inmediata posible. Para ello, se
realizan pequeños prototipos de una funcionalidad que más adelante son
integradas en el proyecto.

La estrategia seguida durante el proyecto puede observarse en la Figura
\ref{fig:estrategia}. Inicialmente se reúnen el director del proyecto con
los desarrolladores para reunir los requisitos necesarios y se realiza un
diseño de las funcionalidades a incluir, se producen varias iteraciones hasta
que todos los requisitos son satisfechos. Una vez finalizado el diseño se pasa
al desarrollo, el cual incluye depuración y pruebas unitarias; el desarrollo
es parte de los desarrolladores, pero la validación debe ser realizada por el
director del proyecto cuando se hayan implementado todos los requisitos de la
iteración en ejecución. Si se han descubierto fallos en el software, es
reparado y si se decide que es requerido ampliar la plataforma se vuelve a
iniciar el ciclo.

\begin{figure}[t]
	\begin{center}
		\includegraphics[scale=0.4]{diagrama-estrategia}
		\caption{Estrategia del desarrollo}
		\label{fig:estrategia}
	\end{center}
\end{figure}

Cuando la iteración del desarrollo ha finalizado correctamente, se pueden
añadir nuevos requisitos o pasar a pruebas en calle si han sido programadas
con anterioridad con el director del proyecto; algunas pruebas requieren de
adaptar parte del programa al carácter de las pruebas.

La planificación y ejecución de las pruebas deben ser supervisadas por el
director del proyecto, y ejecutadas por los desarrolladores. Una vez
finalizadas las pruebas, si se han descubierto fallos en el software, se debe
realizar una depuración del mismo y si se decide que se requiere ampliar la
plataforma se vuelve a iniciar el ciclo.

Las fases del proyecto se puede dividir en los siguientes bloques:
\begin{itemize}
	\item Recolección de requisitos del sistema y estudio de la tecnología
	necesaria.

	\item Desarrollo de la Nube de conductores.

	\item Desarrollo de la aplicación para ciclistas.

	\item Desarrollo de la \gls{rsu}, \gls{obu} y \gls{hmi}.

	\item Verificación del sistema en la calle.

	\item Estudio de los resultados y generar documentación.
\end{itemize}

Los hitos del proyecto son las distintas pruebas que se han hecho de
funcionalidades diferentes del sistema. Durante estas pruebas se analiza el
rendimiento que se obtiene de la comunicación del sistema, se proponen y añaden
mejoras que serán incluidas en siguientes pruebas.

\section{Condiciones de ejecución}
Los recursos están definidos en el apartado de ''Presupuestos''. El lugar de
trabajo será el departamento Mobility de DeustoTech, en la Universidad de
Deusto. El horario y calendario son los correspondientes a los lugares de
trabajo y según su contrato laboral. Todos los medios materiales, servicios
necesarios e instalaciones son provistos por DeustoTech.

DeustoTech es responsable de los servicios provistos: repositorios para el
almacenamiento del código, máquinas virtuales y el correcto funcionamiento de
todos los sistemas informáticos empleados en sus oficinas. Los programadores a
su vez, deben emplear los repositorios de código y copias de seguridad para
garantizar la seguridad e integridad del proyecto.

Los desarrolladores y el jefe de proyecto se comunicarán directamente entre
ellos; a través del correo electrónico o presencialmente. Aunque todos los
requisitos y decisiones deben quedar formalizadas en reunión y reflejadas en
el sistema de gestión de actividades empleado en el departamento: Trello. Los
cambios en algún requisito o la inclusión de una nueva funcionalidad debe
realizarse al finalizar una iteración en ejecución y antes de comenzar otra.
Es decir, no se parará la implementación de una función si se ha iniciado
su desarrollo, aunque se pueden planear nuevos requisitos para futuras
iteraciones.

El proceso para realizar una modificación, o alguna actividad que impliquen
cambios en las especificaciones, diseños o desarrollos realizados, serán
presentados para valoración al equipo del proyecto. En caso de aceptación se
harán las modificaciones pertinentes en el presupuesto y el plan de trabajo.

La aprobación de cada producto lo dará el cliente (el director del proyecto) en
base a las especificaciones que se han acordado en las reuniones durante las
fases de requisitos. La validación del proyecto se realizará cuando se hayan
completado todos los componentes de la solución final, comprobando que se han
cumplido tanto los requisitos como los objetivos que han sido planificados.
Tras la prueba final de la plataforma, el director del proyecto decidirá cómo
documentar la memoria final, además de los artículos o trabajos derivados que
puedan realizarse tras la finalización del presente proyecto.
