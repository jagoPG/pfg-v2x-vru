\section{Prueba 4: grabación de la demostración}
Tras haber validado el funcionamiento de la plataforma completa, se ha
procedido a grabar una demostración del funcionamiento de la plataforma con
carácter publicitario. Las aplicaciones seleccionadas para hacer la demo se
tratan de: los programas para la comunicación de \gls{rsu} y \gls{obu},
aplicación de ciclistas y motoristas, y los elementos auxiliares necesarios
(casco de ciclistas y \gls{hmi}).

Se seguirá el trayecto de un ciclista para una vía pública, y se demostrará el
funcionamiento del sistema de avisos a vehículos y el sistema de aviso en
casco. La duración del vídeo no excederá los 2:00 minutos, y se emplearán
normalmente planos de corta duración desde diferentes perspectivas. Hay que
estudiar el uso de una cámara GoPro para la grabación de planos desde la vista
del vehículo y del ciclista.

\subsection{Recursos materiales}
\begin{itemize}
	\item Máquina virtual con la aplicación del servidor.
	\item 2 módulos Linkbird-MX y sus respectivos transformadores.
	\item 2 módulos \gls{gps}.
	\item Adaptador de corriente para vehículos.
	\item 2 portátiles, uno con la aplicación \gls{obu} y otro con la
	aplicación \gls{rsu}.
	\item Pincho \gls{3g} \gls{usb}.
	\item Casco de ciclistas, con la mota.
	\item Un par de pilas \emph{CR 2032}.
	\item Móvil Android.
	\item Bicicleta.
	\item Cámaras de vídeo.
	\item Trípode.
	\item Cable \gls{usb}.
	\item Lector de tarjetas.
	\item 2 coches.
\end{itemize}

\subsection{Recursos humanos}
Los siguientes recursos son los mínimos necesarios para la realización de
la demostración y la grabación:
\begin{itemize}
	\item 2 personas que dispongan de vehículo.
	\item 1 persona con bicicleta.
	\item 1 persona a la cámara
	\item 1 persona que controle que las aplicaciones funcionen correctamente.
\end{itemize}

\subsection{Emplazamientos seleccionados}
Existen tres posibilidades donde realizar la parte principal de la demo:
\begin{itemize}
	\item Zona de la calle Landaluze (Larrabasterra). Es una vía poco transitada,
	por lo que se puede grabar fácilmente.

	\item Polígono industrial de Galindo (Trapaga).

	\item Pista en Trapagarán.
\end{itemize}

\subsection{Planos}
\begin{enumerate}
	\item Planos cortos:
	\begin{enumerate}
		\item Como se configura la app ciclista.
		\item Visualización de la app del vehículo.
		\item Funcionamiento de la app.
		\item Vista del casco.
		\item Funcionamiento de leds casco.
		\item Vista de las alarmas.
		\item Reacción del vehículo ante la detección.
		\item Reacción del ciclista ante la detección.
	\end{enumerate}

	\item Planos medios:
	\begin{enumerate}
		\item Presentación del ciclista.
		\item Cruce entre vehículos.
	\end{enumerate}

	\item Planos largos:
	\begin{enumerate}
		\item Vista del vehículo.
		\item Seguimiento de ruta del ciclista.
	\end{enumerate}
\end{enumerate}
(*): cambios de plano.

El ciclista sale de casa (*), se pone el casco (*) e inicia la aplicación
móvil (*). Seguidamente, sube a la bicicleta y se marcha (*). Por otro lado,
un conductor se sube al vehículo (*), inicia su aplicación (*) y se pone en
marcha (*). Grabar al ciclista en movimiento (*), y en pantalla dividida se
muestra la aplicación del móvil y cómo funciona el seguimiento de ruta (*).
Cambio de plano al vehículo(*), y mostrar la detección del ciclista (*).

Mostrar cómo reacciona el casco de ciclista (*) y la aplicación, finalmente
ver cómo se cruzan ambos vehículos por su camino correspondiente (*).
