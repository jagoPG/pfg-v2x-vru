\section{Posición vehicular relativa}
Bluetooth Low Energy. A diferencia del clásico Bluetooth, está diseñado para consumir una cantidad significativamente más pequeña de energía. Algunas diferencias durante el desarrollo respecto Bluetooth estándar a tener en cuenta en BLE son:
\begin{enumerate}
	\item El dispositivo está continuamente durmiendo y despertándose para ahorrar batería.
	\item La cantidad de información transmitida es pequeña, como máximo 216 bytes.
	\item La transmisión de información se hace de manera rápida para poder poner el dispositivo a dormir tan pronto como se haya terminado de transmitir la información; latencias de hasta 2 ms por ráfaga.
\end{enumerate}

\subsection{GATT}
Generic Attribute Profile. Establece cómo se va a transmitir la información sobre los perfiles y datos en la conexión BLE. GATT emplea el ATT (Attribute Protocol) como protocolo de transporte para intercambiar datos entre los dispositivos. Los datos están organizados jerárquicamente en secciones llamadas ''servicios'', los cuales tienen piezas relacionadas con ellos denomienadas ''características''.

\subsection{UUID}
Universal Unique Identifier. Es un identificador de 128 bits que está garantizado que es único. Los UUID nos permiten identificar los servicios y características, además de poder operar con ellos.