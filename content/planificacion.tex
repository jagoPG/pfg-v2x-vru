\chapter{Planificación}
En la figura \ref{fig:edt} se puede observar el EDT que ha sido seguido para la
realización de este proyecto. Se ha estimado la realización del proyecto en un
plazo inicial de 2 años, comenzando en Octubre, 2014 y finalizando en Octubre de
2016; se han distribuido las tareas como puede observarse en el diagrama de
Gantt ilustrado en la Figura \ref{fig:gantt}. Para el desarrollo de cada
producto se sigue una metodología de desarrollo iterativa e incremental. Al
final de cada iteración se genera un producto intermedio, con las
características acordadas durante el que funcione correctamente con el resto de
productos.  A continuación se detallan las fases principales del proyecto:
\begin{itemize}
	\item T1 - Análisis de requisitos: recolección y análisis de requisitos.
	Organización de requisitos funcionales y no funcionales.

	\item T2 - Desarrollo de la Nube de conductores: incluye todas las actividades
	para el	desarrollo de la aplicación de conductores que será desplegada en la
	nube.

	\item T3 - Desarrollo de la aplicación de ciclistas: actividades para el
	desarrollo 	de la aplicación de ciclistas y el casco \gls{ble}.

	\item T4 - Desarrollo de las aplicaciones vehiculares: actividades para el
	desarrollo de la \gls{rsu}, \gls{obu} y \glossary{hmi}.

	\item T5 - Validación: pruebas que se han desarrollado en la calle para
	validar diferentes partes del proyecto, y grabación de una demo del
	funcionamiento del proyecto completo.

	\item T6 - Estudios adjuntos: estudios realizados para el desarrollo de
	diferentes áreas del proyecto.
\end{itemize}

\begin{figure}[t]
	\begin{center}
		\rotatebox{90} {
			\includegraphics[scale=0.35]{EDT}
		}
		\caption{Diagrama de desglose de trabajo}
		\label{fig:edt}
	\end{center}
\end{figure}

\begin{figure}[t]
	\rotatebox{90} {
		\includegraphics[scale=0.6]{DiagramaGantt}
	}
	\caption{Diagrama de Gantt}
	\label{fig:gantt}
\end{figure}

Las iteraciones se realizan dentro de las tareas T2, T3 y T4. En cada uno del
las iteraciones se realiza cada una de las siguientes actividades:
\begin{itemize}
	\item Diseño: selección de las especificaciones a implementar y realización
	del diseño de las funciones a implementar en la aplicación.

	\item Desarrollo y depuración: desarrollo y depuración de las funciones
	diseñadas.

	\item Documentación: generación de la documentación de toda la fase
	(comentarios en código, documentos explicativos, prototipos generados...).
\end{itemize}
\FloatBarrier
\section{Paquetes de trabajo}
\begin{table}[ht]
	\centering
	\caption{T1 - Análisis de requisitos}
	\begin{tabular}{ll}
		\toprule
		\multicolumn{2}{c}{\textbf{T1 - Análisis de requisitos}} \\
		\midrule
		\textbf{Duración} & 4 días \\
		\midrule
		\textbf{Objetivos} &
		\begin{tabular}{p{0.8\textwidth}}
			Recolección y análisis de requisitos. Organización de requisitos
			funcionales y no funcionales.
		\end{tabular} \\
		\midrule
		\textbf{Descripción} &
		 \begin{tabular}{p{0.8\textwidth}}
		 	\begin{itemize}
		 		\item Iteración 1: implementación de la arquitectura básica,base de
				datos y acceso a través de sockets.

		 		\item Iteración 2: mejora del formato de mensajes utilizado en la
				comunicación, cambio del acceso por sockets a servlets, e inclusión de
				algoritmos de predicción de accidentes.

		 		\item Iteración 3: cambiada la comunicación con la aplicación de
				ciclistas a \gls{gcm}.

		 		\item Iteración 4: optimización de la plataforma.
		 	\end{itemize}
		 \end{tabular} \\
		\bottomrule
	\end{tabular}
\end{table}

\begin{table}[ht]
	\centering
	\caption{T2 - Desarrollo de la Nube de conductores}
	\begin{tabular}{ll}
		\toprule
		\multicolumn{2}{c}{\textbf{T2 - Desarrollo de la Nube de conductores}} \\
		\midrule
		\textbf{Duración} & 102 días \\
		\midrule
		\textbf{Objetivos} &
		\begin{tabular}{p{0.8\textwidth}}
			Diseño, desarrollo, depuración y despliegue en la nube de la aplicación
			''Nube de conductores''.
		\end{tabular} \\
		\midrule
		\textbf{Descripción} &
		\begin{tabular}{p{0.8\textwidth}}
			\begin{itemize}
				\item Iteración 1: implementación de la arquitectura básica,
				base de datos y acceso a través de sockets.

				\item Iteración 2: mejora del formato de mensajes utilizado en la
				comunicación, cambio del acceso por sockets a servlets, e inclusión de
				algoritmos de predicción de accidentes.

				\item Iteración 3: cambiada la comunicación con la aplicación de
				ciclistas a \gls{gcm}.

				\item Iteración 4: optimización de la plataforma.
			\end{itemize}
		\end{tabular} \\
		\bottomrule
	\end{tabular}
\end{table}

\begin{table}[ht]
	\centering
	\caption{T3 - Desarrollo de la aplicación de ciclistas}
	\begin{tabular}{ll}
		\toprule
		\multicolumn{2}{c}{\textbf{T3 - Desarrollo de la aplicación de ciclistas}}\\
		\midrule
		\textbf{Duración} & 94 días \\
		\midrule
		\textbf{Objetivos} &
		\begin{tabular}{p{0.8\textwidth}}
			Diseño, desarrollo, depuración de la aplicación para ciclistas y el casco
			de seguridad \gls{ble}.
		\end{tabular} \\
		\midrule
		\textbf{Descripción} &
		\begin{tabular}{p{0.8\textwidth}}
			\begin{itemize}
				\item Iteración 1: desarrollo de la base de la aplicación; incluye
				salidas individuales y en grupo.

				\item Iteración 2: mejora de la sensibilidad del \gls{gps}, prueba con
				mensajes \gls{udp} en salidas en grupo, cambio de comunicación a la nube
				a través de mensajes \Gls{http/1.1} y soporte a versiones antiguas de
				Android.

				\item Iteración 3: empleo de \gls{gcm} para la recepción de mensajes y
				desarrollo del casco \gls{ble}.
			\end{itemize}
		\end{tabular} \\
		\bottomrule
	\end{tabular}
\end{table}

\begin{table}[ht]
	\centering
	\caption{T4 - Desarrollo de las aplicaciones vehiculares}
	\begin{tabular}{ll}
		\toprule
		\multicolumn{2}{c}{\textbf{T4 - Desarrollo de las aplicaciones vehiculares}}\\
		\midrule
		\textbf{Duración} & 65 días \\
		\midrule
		\textbf{Objetivos} &
		\begin{tabular}{p{0.8\textwidth}}
			Diseño, desarrollo, depuración y despliegue de las tres aplicaciones para
			vehículos: \gls{obu}, \gls{rsu} y \gls{hmi}.
		\end{tabular} \\
		\midrule
		\textbf{Descripción} &
		\begin{tabular}{p{0.8\textwidth}}
			\begin{itemize}
				\item Iteración 1: base de las aplicaciones \gls{obu} y \gls{rsu}.

				\item Iteración 2: cambio de comunicación a servlets y mejora de los
				mensajes empleados en la comunicación.

				\item Iteración 3: optimización de la plataforma e implementación de la
				aplicación para el \gls{hmi}.
			\end{itemize}
		\end{tabular} \\
		\bottomrule
	\end{tabular}
\end{table}

\begin{table}[ht]
	\centering
	\caption{T5 - Validación}
	\begin{tabular}{ll}
		\toprule
		\multicolumn{2}{c}{\textbf{T5 - Validación}} \\
		\midrule
		\textbf{Duración} & 94 días \\
		\midrule
		\textbf{Objetivos} &
		\begin{tabular}{p{0.8\textwidth}}
			Durante cada iteración de la validación se desarrollan tres actividades:
			una primera para planificar qué pruebas se desean hacer y qué datos se
			desean obtener, así como preparar la aplicación para pruebas. Durante la
			segunda actividad se desarrollan las pruebas. Y finalmente, la tercera
			actividad, se recogen y estudian los resultados y se proponen mejoras
			para futuros desarrollos. Para realizar las estadísticas de los datos
			obtenidos durante las pruebas se ha empleado la librería \emph{matplotlib}
			de \emph{Python}.
		\end{tabular} \\
		\midrule
		\textbf{Descripción} &
		\begin{tabular}{p{0.8\textwidth}}
			\begin{itemize}
				\item Iteración 1: pruebas de rendimiento en comunicaciones \gls{v2x}.

				\item Iteración 2: comunicación \gls{v2v} en entornos urbanos.

				\item Iteración 3: prueba del sistema completo.

				\item Grabación de la demo:
				\begin{itemize}
					\item Planificación: Requisitos generales para la grabación:
					selección de personas que ayuden durante el rodaje, selección de
					escenas y escenarios, materiales, fechas...

					\item Grabación.

					\item Montaje: montaje del vídeo y post-producción.
				\end{itemize}
			\end{itemize}
		\end{tabular} \\
		\bottomrule
	\end{tabular}
\end{table}

\begin{table}[th]
	\centering
	\caption{T6 - Estudios adjuntos}
	\begin{tabular}{ll}
		\toprule
		\multicolumn{2}{c}{\textbf{T6 - Estudios adjuntos}} \\
		\midrule
		\textbf{Duración} & 32 días \\
		\midrule
		\textbf{Objetivos} &
		\begin{tabular}{p{0.8\textwidth}}
			Documentar los estudios realizados durante el desarrollo de las diferentes
			áreas del proyecto para ser publicadas o para facilitar futuros
			desarrollos.
		\end{tabular} \\
		\midrule
		\textbf{Descripción} &
		\begin{tabular}{p{0.8\textwidth}}
			\begin{itemize}
			\item Estudio de OpenXC: se desarrolla un estudio sobre la plataforma
			OpenXC y la posibilidad de su uso en el proyecto.

			\item Estado del arte de Unificación de SDK: se estudia la creación de un
			subproyecto para	desarrollar aplicaciones en el \gls{hmi} de los
			vehículos en	diferentes plataformas; AndroidAuto, CarPlay...

			\item Paper para sensors: desarrollo de un artículo para MDPI sobre
			comunicaciones vehiculares y su uso sobre \gls{vru}.
			\end{itemize}
		\end{tabular} \\
		\bottomrule
	\end{tabular}
\end{table}
