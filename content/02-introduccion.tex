\chapter{Introducción}
Según la Directiva General de la Comisión Europea para el Transporte y la Movilidad,
en 2014, algo más de 25.700 accidentes de tráfico fueron informados en la Unión
Europea. Aunque el número de accidentes se reduce sustancialmente, el informe sobre
Transporte ha anunciado un objetivo estratégico para la seguridad en las carreteras
europeas para el período de 2011 a 2020: reducir el número de muertes en carretera a
la mitad. Si las estadísticas son analizadas, en el período de 2010 a 2012, el número
de ciclistas muertos en siniestros ha aumentado un 6\%; siendo el único agente de la
carretera cuyos resultados vayan a peor. Esto se explica, al menos parcialmente, por un
aumento de la presencia ciclista en la carretera. Se podría decir que el ciclismo es un
medio de transporte donde los \gls{vru} tienen un mayor contacto con el tráfico de
mayor afluencia y velocidad. Cuando están involucrados en un accidente, son los que sufren
las consecuencias más graves derivadas de una colisión con otro agente de la carretera; ya
que están completamente expuestos a otros vehículos.

Basados en estos estudios, los accidentes en la que están involucrados \gls{vru}s ocurren
frecuentemente en vías diseñadas para viandantes y ciclistas; por ejemplo en pasos de
peatones y caminos para ciclistas cercanos a infraestructuras comunes de tráfico, las
carreteras. Por lo tanto, la pregunta es: ¿Cómo se pueden reducir los accidentes de 
\gls{vru}s, y cómo minimizar la gravedad de un siniestro y sus consecuencias? Se pueden tomar
varias soluciones: mejorar el diseño y trazado de las vías de comunicación, mejorar la
iluminación, instalar más infraestructuras de protección, promocionar equipamiento de seguridad
y enseñar cómo utilizarlo...

Sin embargo, hay otras soluciones viables aparte del re-diseño de las infraestructuras
existentes, o soluciones pasivas como el uso del casco de seguridad. Una opción que está
ganando fuerza en los países desarrollados es el desarrollo de soluciones para la movilidad
en el entorno de ciudades inteligentes. Aunque el término de ciudad inteligente pueda parecer
confuso, se podría decir que se considera \emph{inteligente} cuando se ha aplicado tecnologías
de la información y comunicación para mejorar la calidad de vida en áreas como la seguridad,
gasto energético, reducción de costes, y gobierno y transporte, permitiendo una participación
efectiva y activa por parte de los ciudadanos.

En el dominio de las Ciudades Inteligentes, las soluciones para transporte diseñadas tratan
de hacer un uso más seguro, sostenible y eficiente de la carretera a través de un mejor
entendimiento del estado de tráfico, la posición de los los vehículos y usuarios, y el
registro de eventos que suceden durante el transporte. Estas soluciones combinan la capacidad
y beneficios de los sensores, dispositivos, infraestructura física y arquitecturas de
comunicación combinada con sistemas de información en la nube, y la capacidad de analizar
grandes volúmenes de datos.

En este contexto, los Sistemas Inteligentes de Información (ITS) emergen como una respuesta
tecnológica para una mejor motorización y caracterización del tráfico. Estos sistemas
permiten al mismo tiempo mejorar el uso y eficiencia de la carretera, así como la seguridad
de los usuarios, particularmente aquellos definidos como vulnerables; ciclistas, peatones o
motoristas. Los ITS actuales requieren el uso de cámaras de tráfico, paneles informativos,
o sensores de inducción que obtengan datos para ser posteriormente mandados y procesados en
la central de gestión de tráfico. A diferencia de estas soluciones que requieren el uso de
sensores, actualmente surgen sistemas conocidos como Información Vehicular Flotante (FCD),
que se encargan de reunir información de los sistemas de posicionamiento global (GPS) obtenidos
de terminales móviles y el uso de páginas web colaborativas como Waze, que permite a los
conductores obtener y proveer información sobre la carretera sin necesidad de ningún sensor
en la carretera. Este tipo de soluciones basadas en FCD tienen a favor la motorización del
estado de los usuarios de manera ubicua, pero su fiabilidad depende del número de vehículos y
usuarios informando sobre los eventos y aportando datos.

En el dominio de los ITS, los ITS Cooperativos (C-ITS) son sistemas que permiten la conexión
directa entre vehículos (comunicaciones V2V) ó entre vehículos e infraestructuras
(comunicaciones V2I) para intercambiar de información con el objetivo de mejorar la seguridad
vial y la gestión del tráfico. Estos enlaces son posibles gracias a las \gls{obu}, dispositivos
C-ITS dedicados que habilitan interfaces de comunicación, y dispositivos localizados en
infraestructuras llamados \gls{rsu}.

En un escenario C-ITS, hay generalmente cuatro agentes a considerar: dos entidades
móviles (OBUs y peatones), y dos entidades estacionarias (la RSU y el sistema central).
Estas entidades son capaces de ejecutar cuatro tipos diferentes de aplicaciones:
seguridad activa en la carretera, tráfico eficaz cooperativo, servicios locales
cooperativos, y servicios globales en Internet. Sobre cada tipo, hay diferentes
definiciones de casos de uso y aplicaciones, donde cada agente puede ser considerado
un sensor que genera información. Dependiendo de la aplicación y las restricciones de 
tiempo, el intercambio de información entre las entidades se puede clasificar como:
\begin{description}
	\item{Mensajes de alerta:} se definen como notificaciones descentralizadas 
	y pueden ser enviados desde cada vehículo o RSU.

	\item{Mensajes de latido o \emph{"beacons"}:} son usados por los OBU para
	notificar su posición, velocidad e identidad a las RSU que forman parte del
	FCD. Además, estos mensajes también son usados para conocer la situación
	actual del tráfico. Por ello, el Acceso Inalámbrico en Entornos Vehiculares (WAVE)
	define los Mensajes de Aviso Cooperativo (CAMs), que son transmitidos periódicamente
	a todos los vehículos en área de alcance.

	\item{Mensajes sobre infotainment}: notificaciones no relacionadas con la seguridad,
	sino que son usados para aportar mayor información y confort al conductor; datos
	turísticos, acceso a Internet, asistencia en navegación, etc.
\end{description}

En el campo de los servicios C-ITS, una gran variedad de aplicaciones y casos de uso
se centran en incrementar la seguridad del usuario. Teniendo en cuenta requisitos
estratégicos, económicos y de organización, características del sistema así como
requisitos legales y de estandarización, el Comité del Instituto Técnico Europeo de
Estándares en la Telecomunicación ha definido un conjunto básico de aplicaciones para
usar como referencia en ITS para desarrolladores. Entre ellos, los Avisos a Usuarios
Vulnerables en Carretera tratan de proveer notificaciones a los vehículos sobre la
presencia de usuarios vulnerables, por ejemplo ciclistas, y en caso de existir situaciones
de peligro también se avisa a los VRU sobre la presencia de un vehículo cercano.

Siguiendo los requisitos presentados por el ETSI, este proyecto presenta un sistema
que emplea a los vehículos y ciclistas como sensores móviles que aportan información
sobre su posición, velocidad y rumbo con el objetivo de detectar la proximidad entre
estas dos entidades y avisarles en el caso de detectar peligro. Esta solución tiene un
sistema centralizado que despliega comunicación inalámbrica vehicular, conectividad
móvil y computación en la nube, y gestiona la información obtenida por los usuarios
(vehículos y ciclistas). El sistema ha sido desplegado y verificado en un dominio real,
y se han realizado pruebas de rendimiento en diferentes escenarios para comprobar el
correcto funcionamiento de las comunicaciones en los peores escenarios.

Actualmente, existe un proyecto llamado Detección de agentes Inteligentes Cooperativos
para la mejora de la eficiencia en el tráfico (ICSI), con similares aplicaciones que
se encuentra bajo desarrollo usando una solución descentralizada con la finalidad de
mejorar el rendimiento y la seguridad.

\section{Antecedentes}\label{section:antecedentes}
La importancia de las tecnologías C-ITS para la administración pública y la Comisión
Europea está reflejada en la directiva 2010/40/EU, donde la UE reconoce la capacidad
de los C-ITS para mejorar los sistemas de gestión de tráfico actuales y de dirigir
los procesos de implantación y despliegue de estos sistemas en las carreteras
europeas. Tras decenas de investigaciones y proyectos de desarrollo tales como
Sistemas Cooperativos Vehículo-Infraestructura (CVIS), Sistemas Cooperativos para la
seguridad en carretera \emph{\"vehículos inteligentes en ciudades inteligentes\"}
(SAFESPOT), El transporte flexible y adaptable del mañana (TEAM), el despliegue
masivo de sistemas C-ITS se está acercando. Un ejemplo es el Memorandum del
Entendimiento (MOU) firmado por la industria automovilística y organizaciones
constructoras con el objetivo de comenzar a desplegar soluciones basadas en C-ITS en
2015. Las administraciones públicas también están trabajando en la misma dirección,
resaltando el tratado que han pactado Alemania, Austria y Holanda para desplegar sistemas
C-ITS en las vías que comunican estos tres países.

Se puede destacar de la misma manera el anuncio realizado en Febrero del 2014 por la
Administración Nacional de Seguridad del Tráfico en Autopistas (NHTSA), que pertenece
al Departamento de Transporte de los Estados Unidos (USDOT), sus intenciones para dar
los pasos necesarios para el despliegue de sistemas V2V cooperativos en los próximos
años, exactamente a partir de 2017, para vehículos comerciales.
